\section{Conventions for All API}

 {\parindent0pt

  Every HTTP responses from back-end server and the requests from front-end shall follow the conventions presented below.

  \subsection*{Conventions for HTTP response format}

  \begin{itemize}
	  \item[1] Response shall format with 'application/json' content-type;
	  \item[2] Response json shall contain a property "code";
	  \item[3] The value of the "code" shall follow those rules:
		  \begin{itemize}
			  \item Integer type;
			  \item '0' stands for the request is succeed in both logical and business aspect;
			  \item Other codes stands for the request is failed;
			  \item Specific code shall be explained in every FRS;
		  \end{itemize}
  \end{itemize}

  \subsection*{Conventions for API Auth.}

  The system should follow the concept of JWT for HTTP request\&response authorization.
  For Python platform, module \href{https://github.com/jpadilla/pyjwt}{PyJWT} is recommended.

  \begin{itemize}
	  \item[1] The token shall be generated after the \textit{\textbf{login}} request is in code '0';
	  \item[2] The token shall be placed in the HTTP headers with key 'Authorization';
	  \item[3] Every auth-reqired request shall always verify the token first;
	  \item[4] The token shall be expired after the \textit{\textbf{logout}} request or the expire time;
  \end{itemize}

  \subsection*{Other Conventions}

  \begin{itemize}
	  \item[1] API url should follow the fixed context path: \textquote{[host]:[port]/\textbf{scholar-hub}/[request-url]};
	  \item[2] Only \textbf{GET} and \textbf{POST} request are allowed;
	  \item[3] Every data of the \textbf{POST} shall be placed at the \textquote{multipart/form-data} field;
  \end{itemize}
 }

